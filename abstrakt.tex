\documentclass[a4paper, 12pt]{article}

\usepackage{secdot} % Dots in Section Numbers
\usepackage[utf8]{inputenc}
\usepackage[ngerman]{babel}
\usepackage{float}

\usepackage{fancyhdr}

\usepackage{todonotes}
\usepackage{hyperref}

\usepackage[autostyle]{csquotes}
\usepackage[
    backend=biber,
    style=alphabetic,
    sortlocale=de_DE,
    natbib=true,
    url=false,
    doi=true,
    eprint=false
    ]{biblatex}
\addbibresource{literatur.bib}

\usepackage{xparse}
\DeclareDocumentCommand \footATCite{ o o m } {%
  \IfNoValueTF {#2} 
        { \IfNoValueTF {#2}
                {\footnote{\citeauthor{#3}, \emph{\citefield{#3}{shorttitle}}.}}
                {\footnote{#1 \citeauthor{#3}, \emph{\citefield{#3}{shorttitle}}.}} }
        { \IfNoValueTF {#1} 
                    {\footnote{\citeauthor{#3}, \emph{\citefield{#3}{shorttitle}}, S. #2.}}
                    {\footnote{#1 \citeauthor{#3}, \emph{\citefield{#3}{shorttitle}}, S. #2.}} }
  }


\pagestyle{fancy}
\fancyhf{}
\lhead{Jan van Dick}
\chead{Audioadventure - Das Erdbeben aus Chili}
\rhead{\thepage}

\title{%
    Audioadventure - Das Erdbeben aus Chili \\
    \large Nach der Vorlage Heinrich von Kleists \textit{Das Erdbeben in Chili}}
\author{Jan van Dick}
\date{\today}

\begin{document}

\maketitle

\section{Abstrakt}
\textit{Das Erdbeben aus Chili} basiert auf dem Drama Heinrich von Kleists (1807). 
Die veränderte Form der Inszinierung verlangt ebenfalls eine Veränderung des Textes. 
In \textit{Das Erdbeben aus Chili} erleben die Teilnehmer*innen das Stück Kleists aus unterschiedlichen Perspektiven, die sie selbst im Laufe der Inszinierung (mit-)entwickeln.
Dabei wird die Antagonistische Struktur Kleists Drama aufgegriffen und weiterentwickelt. 
Neben dem Thema Liebe und Flucht, werden auf inhaltlicher Ebene die Themen Ideologie und Massenpsychologie aufgegriffen. 
Das persönliche Schicksal Jeronimos und Josephes wird dabei auf eine allgemeine Ebene erhoben.
Die Inszinierung ist dabei inspiriert von Willi Pramels Inszinierung in den Naxus Hallen aus dem Jahre 2016.

\subsection{Audioadventure}
Das Konzept des \glqq Audioadventures\grqq{} soll einen Ansatz darstellen die Idee von Edu-Larp (sog. Educational Liveactionroleplaying) auf Theaterformen zu übertragen. 
Dabei wird das Konzept des \textit{Audiowalks} durch Möglichkeiten des eigenen Interagierens und Intervenierens ergänzt.
Die Zuschauer*innen (oder besser: Teilnehmer*innen) nehmen das Geschehene nicht wie in klassischen Bühnenstücken visuell, sondern über eine Tonspur wahr und werden dadurch selbst zu Agenten des Geschehens, in dem sie durch eine \glqq Stimme in ihrem Kopf\grqq{} angeleitet werden.
Im Gegensatz zu einem \glqq klassischem\grqq{} Audiowalk können die Teilnehmer*innen allerdings stärker auf das Geschehen einwirken und dieses verändern. 
Die Handlungen der Teilnehmer*innen wirken sich direkt auf ihre eigene Tonspur aus und können evtl. auch die der anderen Teilnehmer*innen beeinflussen.
Durch die Relevanz der eigenen Handlungen wird der eigentlich rein künstlerische Aspekt durch den eines Spiels ergänzt.
In der alternativen Realität die in diesem Setting errichtet wird, erhalten die Teilnehmer*innen die Möglichkeit in einem geschützten Rahmen Handlungen zu erproben und bekommen Reaktion durch die Tonspuren und andere Teilnehmer*innen direkt gespiegelt.
Durch die Möglichkeit der (Neu-)Konstruktion von Wirklichkeit können bestehenden Muster der Teilnehmer*innen dekonstruiert werden.
In diesem Sinne greift das Konzept des Audioadventures die Grundlegenden Aspekte auf, die Edu-Larp für Bildungsarbeit produktiv macht.
Durch den (im Gegensatz zum Edu-Larp) stärkeren künstlerischen Aspekt kann die Bildungsarbeit allerdings auf mehreren verschiedenen Ebenen stattfinden: auf der des Spiels und der der Kunst. 

\subsection{Umsetzung}
Die Teilnehmer*innen bekommen zu Beginn des Stückes eine url zu ihrer anfänglichen Tonspur gegeben, die sie öffnen sollen.
Das Stück beginnt gemeinsam in einem dunklen Raum.
Die \hyperref[einleitung]{Einleitung} hören die Teilnehmer*innen gemeinsam über die Anlage.
Am Ende der Einleitung werden sie dazu aufgefordert ihre Kopfhörer aufzusetzen.

Die \glqq Stimme im Ohr\grqq{} erzählt auf drei Ebenen:
\begin{itemize}
    \item[1.] Sie erzählt die Geschichte.
    \item[2.] Sie leitet die Teilnehmer*innen in der \glqq realen Welt\grqq{} durch Räume und Wege entlang, à la: \glqq Verlasse jetzt die Tür auf den Hof\grqq .
    \item[3.] Sie kommentiert das Geschehen auf einer Metaebene.
\end{itemize}
Dabei könnten für die verschiedenen Ebenen verschiedene Stimme eingesetzt werden (wobei es sich anbietet eine Stimme für Ebene 1 und 2 und eine getrennte für die Ebene 3 zu verwenden).\\

Im Laufe des Audioadventures kann sich die Tonspur der Teilnehmer*innen auf zwei Weisen ändern:
\begin{itemize}
    \item[1.] Ein*e Teilnehmer*in kann ihre Tonspur selbstständig wechseln, indem sie z.B. ihren Weg verlässt (was sich im Weiteren als ein Nicht-Folgen ergeben wird) und eine neue Tonspur findet (neue url).
    \item[2.] Der Server kann zählen, wie viele Teilnehmer*innen in bestimmten Tonspuren \glqq eingehängt\grqq{} sind. Je nachdem, wie viele Teilnehmer*innen alternative Wege gegangen sind, können sich dadurch die anderen Tonspuren ebenfalls ändern.
\end{itemize}
Hierzu genauer im Abschnitt \hyperref[technische_umsetzung]{Technische Umsetzung}.


\section{Das Erdbeben aus Chili}

\subsection{Einleitung}\label{einleitung}
\textit{Die Teilnehmer*innen begeben sich in einen spärlich ausgeleuchteten Raum voller Nebel. 
Auf dem Boden liegen blutbeschmierte, menschengroße Puppen und zerrissene Kleider. 
Die Puppen sind mit bunter Farbe bemalt, doch die Farbe ist verschmiert und mit Blut überzogen. Keine Musik, aber Geräusche sind zu hören. 
Das Licht geht aus, die Geräusche verklingen und aus dem Off erklingt eine Frauenstimme.}\\

Der Staub verzieht sich, die Münder sind noch trocken, die Hände noch über unseren Köpfen zusammengeschlagen, bald kommt die Nacht und das Mondscheinlicht, doch derweilen können wir einander noch sehen. 
Bald sind die Stimmen schreiender Mütter und weinender Kinder von ihrem Körper entbunden, wir fühlen sie dann nicht mehr?
Der Staub zerzieht sich, die Münder sind noch trocken. \\

Fühlst du mich? 
Bin ich noch zu leer? 
Ich bin zunächst eine Stimme. 
Ich existiere nur in deinem Ohr, körperlos.
Versuchst du mich zu greifen oder dich an mir festzuhalten, musst du bald feststellen, dass es nichts Greifbares gibt, nur Wellen von der jede einzelne schon wieder verschwunden ist, wenn sie sich zum Wort verwandelt hat;
der Ton zerrinnt zwischen deinen Fingern und nur, dass du mich hörst, ist der Beweis, dass ich doch irgendwie bin.\\

Der Staub zerzieht sich, die Münder sind noch trocken, die Stimmen der Schreienden sind laut, ihre Gesichter verzehrt, dreckig und blutig rinnen ihnen Tränen über ihr Gesicht.
Bald ist es dunkel, dann kommt das Mondscheinlicht, dann sehen wir sie (endlich) nicht mehr.
Aber von dem Staub sehen wir auch kaum zehn Meter weit. 
Er ist überall, dieser grausame Staub, doch langsam verzieht er sich. \\

Hörst du mich? Willst du mich hineinlassen? In dich? Eine Fremde? Eine bloße Stimme?
Bin ich dir noch zu leer?
Aber ich muss nicht Stimme bleiben. 
Damit greife ich wohl einen Gedanken Nietzsches auf. 
Dass ich nicht Stimme bleiben muss, scheint mir das zu sein, was Nietzsche beschreibt, wenn er erklärt, dass wir einen Baum nie vollständig sehen, ihn eher aus einigen Ästen, Farbeindrücken und Blättern uns erdichten.
Oder, dass wir nie alle Wörter lesen, sondern hier und dort uns eines herauspicken, um uns das Drumherum zu erschließen.\\

Wir sind eng aneinander gedrängt.
Unsere Haut berührt sich hier und dort.
Einige von uns halten sich an den Händen.
Die Stadt ist ein Trümmerfeld geworden. 
Ein Trümmerfeld aus Schreien.
Ein Trümmerfeld aus Verzweiflung.
Wen werden die Leichen nicht noch ewig auf den Augen brennen?\\

Fühlst du mich? Werde ich dir langsam warm? Nietzsche schreibt: 
\glqq Wir erdichten uns den größten Teil des Erlebnisses und sind kaum dazu zu zwingen nicht als \glq Erfinder\grq{} irgendeinem Vorgang zuzuschauen. 
Dies alles will sagen: Wir sind von Grund aus, von alters her \textit{ans Lügen gewöhnt}. Oder, um es tugendhafter und heuchlichericher, kurz: angenehmer auszurücken: 
Man ist viel mehr Künstler, als man weiß\grqq. 
Wie viel dichtest du mir an?\\

Der Staub zerzieht sich. 
Das Blut bleibt.
Der Tag weicht der Nacht
Was gesehen ist, wird bald verwunden seinen.
Doch die Schreie sie bleiben.
Sie bleiben in unseren Ohren. \\

Wie viel dichtest du mir an? 
Mit welchen Metaphern fütterst du den Ton, der durch deinen Kopf schleicht und so Sekunde für Sekunde mehr Körper bekommt, mehr Bild wird und schließlich so echt wirkt, wie ein Traum und nur noch die Frage bleibt: Was unterscheidet Traum und Wirklichkeit? (Oder: bin ich noch weniger wirklich als die Menschen um dich herum?)\\

Jetzt ist die Nacht da. 
Jetzt hat sich der Staub scheinbar noch dichter zugezogen.
Nur wir können uns noch sehen, weil wir einander kennen. 
Es ist jetzt Zeit, jetzt können wir nicht mehr gesehen werden. 
Jetzt los, die Trümmern, die Schreie: hinter uns gelassen. 
Macht schnell!\\

Es hat mich so danach verlangt mit dir zu reden? Wirst du mir folgen nachher? 
Aber lass mich doch noch ein wenig näher an dich heran, indem du dir die Technik zu nutze machst, die sich in deiner Hosentasche versteckt, um mich dann ganz nahe bei deinem Ohr zu habe, dann übernehme ich, als dein Objekt die Führung über dein Erdichten und Erfinden und Künstler-Sein.\\

\subsection{Die erste Entscheidung (Gruppe 1)}
Ist es so besser? Jetzt bin ich nur für dich da.
Jetzt bin ich nur für dich da. Wirst du mir folgen? 
Ist es so besser?
Wenn du mir vertraust, dann verlasse jetzt den Raum durch die große Tür. 
Gehe langsam und vorsichtig. 
Wir wollen nicht gehört oder gesehen werden. 
Verlasse langsam den Raum durch die große Tür.\\
Die Zeit ist gekommen, dass wir uns bewegen und die über unseren Köpfen zusammengeschlagenen Arme sinken lassen und uns nicht mehr verstecken. 
Die Punkt ist gekommen, an dem wir die gefallene Stadt uns zu eigen machen können. 
Wir versammeln uns in dem nächste Raum und warten, bis alle eingetroffen sind.
Wir sind solidarisch, wir bleiben zusammen.
Wartet auf die anderen. Wir haben lange genug gewartet und wollen jetzt nichts überstürzen.
Wenn so eine Stadt zusammenbricht und in die Knie gezwungen wird, ist es für die ewig wartenden, die lauernden manchmal eine Freude jetzt endlich ausbrechen zu können. 
Darauf haben wir gewartet.
Für uns geborenen Räthselrather, ist so manche Sonnenfinsternis, das neue, noch nie so offen gewesene Meer. 
\begin{itemize}

    \item[] \textit{In der Masse, meint Le Bon, vermischen sich die individuellen Erwerbungen des Einzelnen, und damit verschwindet deren Eigenart. 
Das rassenmäßige Unbewusste tritt hervor, das Heterogene versinkt im Homogenen.
Das Individuum erlangt ein Gefühl der unüberwindbaren Macht, welche ihm gestattet Trieben zu frönen, die es sonst notwendigerweise unterdrückt hätte.\footfullcite[][37]{freud_massen_1993}
Die Masse ist außerordentlich leicht beeinflußbar und leichtgläubig, sie ist kritiklos, das Unwahrscheinliche existiert für die nicht.\footATCite[][41]{freud_massen_1993}
Während die intellektuelle Leistung der Masse  immer tief unter der des einzelnen steht, kann ihr ethisches Verhalten dieses Niveau hich überragen, wie tief darunter herabgehen.\footATCite[][42]{freud_massen_1993}}
\end{itemize}

Sind wir beisammen, haben wir uns gefunden? 
Wir müssen uns aufeinander verlassen können. 
Wenn einem etwas auffällt, was verdächtig ist, brauchen wir ein Zeichen.
Wenn einer den Arm hebt, bedeutet das Gefahr und ein jeder drückt sich mit dem Rücken an die nächste Wand. 
Wollen wir es versuchen?


\subsection{Die erste Entscheidung (Gruppe 2)}
\textit{Während die anderen den Raum verlassen haben geht langsam das Licht wieder an und beleuchtet die blutbeschmierten Puppen}

\section{Technische Umsetzung}\label{technische_umsetzung}
   
\printbibliography
 
\end{document}
