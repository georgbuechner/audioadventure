\documentclass[a4paper, 12pt]{report}

\usepackage{secdot} % Dots in Section Numbers
\usepackage[utf8]{inputenc}
\usepackage[ngerman]{babel}
\usepackage{float}

\usepackage{fancyhdr}

\usepackage[textwidth=90, backgroundcolor=blue]{todonotes}
\usepackage{hyperref}

\usepackage{soul}

\usepackage[autostyle]{csquotes}
\usepackage[
    backend=biber,
    style=alphabetic,
    sortlocale=de_DE,
    natbib=true,
    url=false,
    doi=true,
    eprint=false
    ]{biblatex}
\addbibresource{literatur.bib}

\usepackage{xparse}
\DeclareDocumentCommand \footATCite{ o o m } {%
  \IfNoValueTF {#2}% 
        { \IfNoValueTF {#2}%
                {\footnote{\citeauthor{#3}, \emph{\citefield{#3}{shorttitle}}.}}%
                {\footnote{#1 \citeauthor{#3}, \emph{\citefield{#3}{shorttitle}}.}}% 
        }%
        { \IfNoValueTF {#1}%
                    {\footnote{\citeauthor{#3}, \emph{\citefield{#3}{shorttitle}}, S. #2.}}%
                    {\footnote{#1 \citeauthor{#3}, \emph{\citefield{#3}{shorttitle}}, S. #2.}}% 
        }%
  }
\DeclareDocumentCommand \qq{m} {\glqq #1\grqq}

\usepackage{titlesec}
\titleformat{\chapter}[hang]{\Huge\bfseries}{}{0pt}{\Huge\bfseries}
\newcommand\frontmatter{ \cleardoublepage \pagenumbering{roman}}
\newcommand\mainmatter{ \cleardoublepage \pagenumbering{arabic}}
\newcommand\backmatter{ \cleardoublepage \pagenumbering{roman}}


\pagestyle{fancy}
\fancyhf{}
\lhead{Jan van Dick}
\chead{Audioadventure - Das Erdbeben aus Chili}
\rhead{\thepage}

\title{%
    Audioadventure - Das Erdbeben aus Chili \\
    \large Nach der Vorlage Heinrich von Kleists \textit{Das Erdbeben in Chili}}
\author{Jan van Dick}
\date{\today}

\begin{document}

\maketitle
\frontmatter


\chapter*{Abstrakt}
\textit{Das Erdbeben aus Chili} basiert auf dem Drama Heinrich von Kleists (1807). 
Die veränderte Form der Inszinierung verlangt ebenfalls eine Veränderung des Textes. 
In \textit{Das Erdbeben aus Chili} erleben die Teilnehmer*innen das Stück Kleists aus unterschiedlichen Perspektiven, die sie selbst im Laufe der Inszinierung (mit-)entwickeln.
Dabei wird die Antagonistische Struktur Kleists Drama aufgegriffen und weiterentwickelt. 
Neben dem Thema Liebe und Flucht, werden auf inhaltlicher Ebene die Themen Ideologie und Massenpsychologie aufgegriffen. 
Das persönliche Schicksal Jeronimos und Josephes wird dabei auf eine allgemeine Ebene erhoben.
Die Inszinierung ist dabei inspiriert von Willi Pramels Inszinierung in den Naxus Hallen aus dem Jahre 2016.

\section*{Audioadventure}
Das Konzept des \glqq Audioadventures\grqq{} soll einen Ansatz darstellen die Idee von Edu-Larp (sog. Educational Liveactionroleplaying) auf Theaterformen zu übertragen. 
Dabei wird das Konzept des \textit{Audiowalks} durch Möglichkeiten des eigenen Interagierens und Intervenierens ergänzt.
Die Zuschauer*innen (oder besser: Teilnehmer*innen) nehmen das Geschehene nicht wie in klassischen Bühnenstücken visuell, sondern über eine Tonspur wahr und werden dadurch selbst zu Agenten des Geschehens, in dem sie durch eine \glqq Stimme in ihrem Kopf\grqq{} angeleitet werden.
Im Gegensatz zu einem \glqq klassischem\grqq{} Audiowalk können die Teilnehmer*innen allerdings stärker auf das Geschehen einwirken und dieses verändern. 
Die Handlungen der Teilnehmer*innen wirken sich direkt auf ihre eigene Tonspur aus und können evtl. auch die der anderen Teilnehmer*innen beeinflussen.
Durch die Relevanz der eigenen Handlungen wird der eigentlich rein künstlerische Aspekt durch den eines Spiels ergänzt.
In der alternativen Realität die in diesem Setting errichtet wird, erhalten die Teilnehmer*innen die Möglichkeit in einem geschützten Rahmen Handlungen zu erproben und bekommen Reaktion durch die Tonspuren und andere Teilnehmer*innen direkt gespiegelt.
Durch die Möglichkeit der (Neu-)Konstruktion von Wirklichkeit können bestehenden Muster der Teilnehmer*innen dekonstruiert werden.
In diesem Sinne greift das Konzept des Audioadventures die Grundlegenden Aspekte auf, die Edu-Larp für Bildungsarbeit produktiv macht.
Durch den (im Gegensatz zum Edu-Larp) stärkeren künstlerischen Aspekt kann die Bildungsarbeit allerdings auf mehreren verschiedenen Ebenen stattfinden: auf der des Spiels und der der Kunst. 

\section*{Umsetzung}
Die Teilnehmer*innen bekommen zu Beginn des Stückes eine url zu ihrer anfänglichen Tonspur gegeben, die sie öffnen sollen.
Das Stück beginnt gemeinsam in einem dunklen Raum.
Die \hyperref[einleitung]{Einleitung} hören die Teilnehmer*innen gemeinsam über die Anlage.
Am Ende der Einleitung werden sie dazu aufgefordert ihre Kopfhörer aufzusetzen.

Die \glqq Stimme im Ohr\grqq{} erzählt auf drei Ebenen:
\begin{itemize}
    \item[1.] Sie erzählt die Geschichte.
    \item[2.] Sie leitet die Teilnehmer*innen in der \glqq realen Welt\grqq{} durch Räume und Wege entlang, à la: \glqq Verlasse jetzt die Tür auf den Hof\grqq .
    \item[3.] Sie kommentiert das Geschehen auf einer Metaebene.
\end{itemize}
Dabei könnten für die verschiedenen Ebenen verschiedene Stimme eingesetzt werden (wobei es sich anbietet eine Stimme für Ebene 1 und 2 und eine getrennte für die Ebene 3 zu verwenden).\\

Im Laufe des Audioadventures kann sich die Tonspur der Teilnehmer*innen auf zwei Weisen ändern:
\begin{itemize}
    \item[1.] Ein*e Teilnehmer*in kann ihre Tonspur selbstständig wechseln, indem sie z.B. ihren Weg verlässt (was sich im Weiteren als ein Nicht-Folgen ergeben wird) und eine neue Tonspur findet (neue url).
    \item[2.] Der Server kann zählen, wie viele Teilnehmer*innen in bestimmten Tonspuren \glqq eingehängt\grqq{} sind. Je nachdem, wie viele Teilnehmer*innen alternative Wege gegangen sind, können sich dadurch die anderen Tonspuren ebenfalls ändern.
\end{itemize}
Hierzu genauer im Abschnitt \hyperref[technische_umsetzung]{Technische Umsetzung}.

\tableofcontents
\mainmatter
\chapter{Das Erdbeben aus Chili}

\section{Einleitung}\label{einleitung}
\textit{Die Teilnehmer*innen begeben sich in einen spärlich ausgeleuchteten Raum voller Nebel. 
Auf dem Boden liegen blutbeschmierte, menschengroße Puppen und zerrissene Kleider. 
Die Puppen sind mit bunter Farbe bemalt, doch die Farbe ist verschmiert und mit Blut überzogen. Keine Musik, aber Geräusche sind zu hören. 
Das Licht geht aus, die Geräusche verklingen und aus dem Off erklingt eine Frauenstimme.}\\

Der Staub verzieht sich, die Münder sind noch trocken, die Hände noch über unseren Köpfen zusammengeschlagen, bald kommt die Nacht und das Mondscheinlicht, doch derweilen können wir einander noch sehen. 
Bald sind die Stimmen schreiender Mütter und weinender Kinder von ihrem Körper entbunden, wir fühlen sie dann nicht mehr?
Der Staub zerzieht sich, die Münder sind noch trocken. \\

Fühlst du mich? 
Bin ich noch zu leer? 
Ich bin zunächst eine Stimme. 
Ich existiere nur in deinem Ohr, körperlos.
Versuchst du mich zu greifen oder dich an mir festzuhalten, musst du bald feststellen, dass es nichts Greifbares gibt, nur Wellen von der jede einzelne schon wieder verschwunden ist, wenn sie sich zum Wort verwandelt hat;
der Ton zerrinnt zwischen deinen Fingern und nur, dass du mich hörst, ist der Beweis, dass ich doch irgendwie bin.\\

Der Staub zerzieht sich, die Münder sind noch trocken, die Stimmen der Schreienden sind laut, ihre Gesichter verzehrt, dreckig und blutig rinnen ihnen Tränen über ihr Gesicht.
Bald ist es dunkel, dann kommt das Mondscheinlicht, dann sehen wir sie (endlich) nicht mehr.
Aber von dem Staub sehen wir auch kaum zehn Meter weit. 
Er ist überall, dieser grausame Staub, doch langsam verzieht er sich. \\

Hörst du mich? Willst du mich hineinlassen? In dich? Eine Fremde? Eine bloße Stimme?
Bin ich dir noch zu leer?
Aber ich muss nicht Stimme bleiben. 
Damit greife ich wohl einen Gedanken Nietzsches auf. 
Dass ich nicht Stimme bleiben muss, scheint mir das zu sein, was Nietzsche beschreibt, wenn er erklärt, dass wir einen Baum nie vollständig sehen, ihn eher aus einigen Ästen, Farbeindrücken und Blättern uns erdichten.
Oder, dass wir nie alle Wörter lesen, sondern hier und dort uns eines herauspicken, um uns das Drumherum zu erschließen.\\

Wir sind eng aneinander gedrängt.
Unsere Haut berührt sich hier und dort.
Einige von uns halten sich an den Händen.
Die Stadt ist ein Trümmerfeld geworden. 
Ein Trümmerfeld aus Schreien.
Ein Trümmerfeld aus Verzweiflung.
Wen werden die Leichen nicht noch ewig auf den Augen brennen?\\

Fühlst du mich? Werde ich dir langsam warm? Nietzsche schreibt: 
\glqq Wir erdichten uns den größten Teil des Erlebnisses und sind kaum dazu zu zwingen nicht als \glq Erfinder\grq{} irgendeinem Vorgang zuzuschauen. 
Dies alles will sagen: Wir sind von Grund aus, von alters her \textit{ans Lügen gewöhnt}. Oder, um es tugendhafter und heuchlichericher, kurz: angenehmer auszurücken: 
Man ist viel mehr Künstler, als man weiß\grqq\footfullcite[][114]{nietzsche_jenseits_2014}. 
Wie viel dichtest du mir an?\\

Der Staub zerzieht sich. 
Das Blut bleibt.
Der Tag weicht der Nacht
Was gesehen ist, wird bald verwunden seinen.
Doch die Schreie sie bleiben.
Sie bleiben in unseren Ohren. \\

Wie viel dichtest du mir an? 
Mit welchen Metaphern fütterst du den Ton, der durch deinen Kopf schleicht und so Sekunde für Sekunde mehr Körper bekommt, mehr Bild wird und schließlich so echt wirkt, wie ein Traum und nur noch die Frage bleibt: Was unterscheidet Traum und Wirklichkeit? (Oder: bin ich noch weniger wirklich als die Menschen um dich herum?)\\

Jetzt ist die Nacht da. 
Jetzt hat sich der Staub scheinbar noch dichter zugezogen.
Nur wir können uns noch sehen, weil wir einander kennen. 
Es ist jetzt Zeit, jetzt können wir nicht mehr gesehen werden. 
Jetzt los, die Trümmern, die Schreie: hinter uns gelassen. 
Macht schnell!\\

Es hat mich so danach verlangt mit dir zu reden? Wirst du mir folgen nachher? 
Aber lass mich doch noch ein wenig näher an dich heran, indem du dir die Technik zu nutze machst, die sich in deiner Hosentasche versteckt, um mich dann ganz nahe bei deinem Ohr zu habe, dann übernehme ich, als dein Objekt die Führung über dein Erdichten und Erfinden und Künstler-Sein.\\

\section{Gruppe 1}
\subsection{Der Weg zur Tribüne I}
Ist es so besser? Jetzt bin ich nur für dich da.
Jetzt bin ich nur für dich da. Wirst du mir folgen? 
Ist es so besser?
Wenn du mir vertraust, dann verlasse jetzt den Raum durch die große Tür. 
Gehe langsam und vorsichtig. 
Wir wollen nicht gehört oder gesehen werden. 
Verlasse langsam den Raum durch die große Tür.\\

Die Zeit ist gekommen, dass wir uns bewegen und die über unseren Köpfen zusammengeschlagenen Arme sinken lassen und uns nicht mehr verstecken. 
Die Punkt ist gekommen, an dem wir die gefallene Stadt uns zu eigen machen können. 
Wir versammeln uns in dem nächste Raum und warten, bis alle eingetroffen sind.
Wir sind solidarisch, wir bleiben zusammen. 
Wartet auf die anderen. Wir sind das Warten und die Gedulde gewohnt und wollen darum jetzt nichts überstürzen.
Wenn so eine Stadt zusammenbricht und in die Knie gezwungen wird, ist es für die ewig Wartenden, die Lauernden manchmal eine Freude jetzt endlich ausbrechen zu können. 
Für uns geborenen Räthselrather, ist so manche Sonnenfinsternis, das neue, noch nie so offen gewesene Meer. 
\begin{itemize}
    \item[] \textit{In der Masse, meint Le Bon, vermischen sich die individuellen Erwerbungen des Einzelnen, und damit verschwindet deren Eigenart. 
Das rassenmäßige Unbewusste tritt hervor, das Heterogene versinkt im Homogenen.
Das Individuum erlangt ein Gefühl der unüberwindbaren Macht, welche ihm gestattet Trieben zu frönen, die es sonst notwendigerweise unterdrückt hätte.\footfullcite[][37]{freud_massen_1993}
Die Masse ist außerordentlich leicht beeinflußbar und leichtgläubig, sie ist kritiklos, das Unwahrscheinliche existiert für die nicht.\footATCite[][41]{freud_massen_1993}
Während die intellektuelle Leistung der Masse  immer tief unter der des einzelnen steht, kann ihr ethisches Verhalten dieses Niveau hich überragen, wie tief darunter herabgehen.\footATCite[][42]{freud_massen_1993}}
\end{itemize}

Sind wir beisammen, haben wir uns gefunden? 
Wir müssen uns aufeinander verlassen können. 
Wenn einem etwas auffällt, was verdächtig ist, brauchen wir ein Zeichen.
Wenn einer den Arm hebt, bedeutet das Gefahr und ein jeder drückt sich mit dem Rücken an die nächste Wand. 
Wollen wir es versuchen?
\textit{(Nach einer kurzen Pause:)} Versuche es nur. \todo[noline]{Check waiting time.} \\

\subsection{Die Gasse}

Verlasse jetzt den Raum und betrete die Straße. 
Atme den Duft ein, er ist staubig und trocken, aber es ist dein Duft, der Duft, der bald dein Sein wird.
Laufe langsam an der Hauswand entlang, wir sind auf dem Weg.
Chili ist eine uralte Stadt, schon lange haben wir geplant sie von ihren Ketten zu befreien. 
Heute sind wir lebendig, wir spüren, dass die Zeit gekommen ist.
Wir müssen jetzt entschlossen gemeinsam gehen und Chili in die Stadt verwandeln, wie wir sie seit je her gesehen haben.
Viel zu lange stand die Stadt unter der Herrschaft der anderen. Deren, die es nicht verdient haben.
Sie hatten erkannt, dass sie schwach sind und haben uns ihre Schwäche aufgedrängt. 
Doch wir meinen: Das Ressentiment muss sterben!\\

Wir passieren eine kleine Gasse, hier treiben sich die Aufständigen herum die gegen den Geist der Stadt ihr eigenes durchzusetzen versuchen.
Sie wollen dem Ressentiment die Hand reichen und mit der anderen noch die letzten vornehmen Züge der Stadt entreißen.
Wenn du magst, zieh, deine Kopfhörer ab und erkunde, was die anderen aus der Krise der Stadt machen wollen, doch sei gewarnt, sie sind nicht auf unserer Seite.
Die dunkle Gasse liegt jetzt hinter uns. 
Zu viele Gassen sind dunkel geworden in letzter Zeit.\todo{Insert a Szene of first confrontation between Gruppe A and C}\\

\subsection{Der Weg zur Tribüne II}
Doch wir laufen weiter geradeaus, um diese Stadt zu befreien.
Laufe weiter gerade aus, an der Gasse vorbei und die Kantstraße entlang. 
Wir können schon die Stimmen der Bewohner hören, die uns zujubeln. 
Manchmal muss für eine Stadt die Katastrophe kommen, damit sie aus ihrer Asche neu entstehen kann.
Und wir sind der Phönix der sie aus der toten Asche an die lebendige Luft bringen wird.
Laufe weiter die Kantstraße entlang.
Der Staub verzieht sich. 
Er macht seinen neuen Herren Platz. 
Er gibt Raum.
Der Absatz unserer Stiefel hallt durch die leeren Straßen, und entfaltet sich, wie ein rasendes Phantom in der Stadt.
Bald werden alle von uns hören.
Einige erwarten schon unsere Ankunft, wie werden sie uns begrüßen?
Wie werden wir sie grüßen und anreden?
Wie grüßt man seine Untertanen? Genossen? Kameraden? Volk? 
Achte darauf, dass noch alle beisammen sind. 
Du trägst die Verantwortung!
Wir kommen zu günstiger Zeit, doch können wir nur als Gruppe stark genug sein.\\

\subsection{Der Händedruck}
\textit{(Die Teilnehmer*innen befinden sich draußen und laufen auf eine Kreuzung zu. 
Auf der Kreuzung beginnen sie diese zu umkreisen und gehen nacheinander zu einem Menschen in der Mitte der Kreuzung.
Sie sind aufgefordert dieser Person die Hand zu geben.
Tun sie es, so erhalten sie eine Notiz mit einem Satz, den sie evtl. später sprechen müssen.
Sagen sie stattdessen: \glqq \textit{Dir} geben ich nicht die Hand\grqq, so erhalten sie einen neuen QR-Code.)}\\

An der nächsten Kreuzung überquere den Sandweg, dann biege links ab, auf die andere Straßenseite.
Siehst du den Menschen in der Mitte der Kreuzung? 
Er ist unsere Kontaktpersonen.
Verhalte dich noch unauffällig.
Überquere jetzt den Sandweg und biege links ab, auf die andere Straßenseite. 
Überquere den Sandweg und dann die Kantstraße.
Wenn du die Kantstraße überquert hast, drehe dich nach links und überquere den Sandweg.
Gehe nun wieder nach links und überquere die Kantstraße, dann, wieder den Sandweg.
Kant. Sand. Kant. Sand.
Während wir so die Kreuzung umkreisen, geht einer nach dem anderen zu der Kontaktperson und gibt ihr die Hand. 
Aber nicht alle auf einmal.
Verhalte dich unauffällig.
Wir müssen unseren eigenen Rhythmus finden.
Wann wirst du gehen, dass du nicht gleichzeitig mit einem anderen gehst?
Achtet darauf nicht gleichzeitig zu gehen!
Wir wollen die neuen Herrscher dieser Stadt sein. 
Als neue Herrscher müssen wir es im Blut spüren, was die anderen tun wollen.
Wir sind ein Organismus. Wir müssen ein Organismus sein!
Wir umkreisen weiter die Kreuzung.
Eine nach der anderen geht zur Kontaktperson.
Wann wirst du gehen?
\begin{itemize}
    \item[] \textit{Die Notwendigkeit der Identität ist die Entwicklung der Idee innerhalb ihrer selbst;\footfullcite[][249]{hegel_grundlinien_2017} Im Lebendigen ist der Zweck in der Materie immanente Bestimmung und Tätigkeit und alle Glieder sind ebenso sich gegenseitig Mittel als Zweck.\footfullcite[][140]{hegel_enzyklopadie_1970} Ihren besonderen Inhalt nimmt die Gesinnung aus den verschiedenen Seiten des Organismus des Staats. Dieser Organismus ist die Entwicklung der Idee zu ihren Unterschieden und zu deren objektiven Wirklichkeit. Diese unterschiedenen Seiten sind so die verschiedenen Gewalten [...] wodurch das Allgemeine sich fortwährend und zwar indem sie durch die Natur des Begriffes bestimmt sind, auf notwendige Weise hervorbringt. [...] - dieser Organismus ist die politische Verfassung.\footATCite[][251]{hegel_grundlinien_2017}}
\end{itemize}

Wir drehen uns erneut nach links und überqueren ein letztes Mal die Kantstraße, dann wieder links und auch ein letztes Mal den Sandweg. 
Danach geht es geradeaus, zur Tribüne! 
Laufe jetzt gerade aus, die Kant Straße zurück. 
Unsere Zeit ist gekommen, wir werden sie uns nehmen!\\

\subsection{Der Weg zur Tribüne III}
Während so der Absatz unserer Stiefel durch die leeren Straßen hallt, hallt durch unseren Kopf schon der Satz:
\glqq Das Parlament ist gefallen. Lang lebe das Parlament, wir sind das neue Parlament!\grqq{}
Einige werden schon auf uns warten, wenn wir ankommen.
Es sind die, die das Parlament von Beginn an nicht leiden konnten.
Was unsere Stadt vor dem war, daran konnten sie sich noch erinnern. 
Sie haben von uns gehört und jetzt fühlen sie ihre eigene Kraft in sich zurückkehren, dass sie diejenigen sind, die gehört werden müssen.
Alle die, die heute kommen, die den alten Geist der Stadt noch in sich tragen werden unsere wahren Untertan sein!
Der Rest, die sich damals auf den Thron setzten, um aus einem demokratischen Gefühl heraus, diejenigen beschützen wollten, die sie selbst waren, wird zusammen mit all den \glqq Neuen Werten\grqq{} untergehen. 
Laufe weiter die Kantstraße hinunter, bald sind wir da, bald wird man uns hören. 
Das durch die Stadt hallende Klopfen unserer Absätze trägt den heiligen Satz vor uns her: 
\glqq Das Parlament ist gefallen. Lang lebe das Parlament!\grqq\\
\begin{itemize}
    \item[] \em
    Den Fingerzeig zum \so{rechten} Wege, gab mir die Frage, was eigentlich die von den verschiedenen Sprachen ausgeprägte Bezeichnung des \qq{Guten} in etymologischer Hinsicht zu bedeuten habe: 
    da fand ich, dass sie allesammt auf die \so{gleiche Begriffs-Verwandlung} zurückleiten, - dass überall \qq{vornehm}, \qq{edel} im ständischen Sinne der Grundbegriff ist, aus dem sich \qq{gut} im Sinne von \qq{seelisch-vornehm}, \qq{edel}, von \qq{seelisch-hochgeartet}, \qq{seelisch-privilegiert} mit Notwendigkeit heraus entwickelt: eine Entwicklung, die immer parallel mit jener anderen läuft, welche \qq{gemein}, \qq{pöbelhaft}, \qq{niedrig}, schließlich in den Begriff \qq{schlecht} übergehen macht. 
    Das beredteste Beispiel für das Letztere ist das deutsche Wort \qq{schlecht} selber: 
    als welches mit \qq{schlicht} identisch ist - vergleiche \qq{schlechtweg}, \qq{schlechterdings} - und ursprünglich den schlichten, den gemeinen Mann noch ohne einen verdächtigen Seitenblick, einfach im Gegensatz zum Vornehmen bezeichnete.\footATCite[][261]{nietzsche_jenseits_2014}
\end{itemize}

Sind wird langsam angekommen? 
Fühlen wir uns sicher in unserem Vorhaben?
Wir sehen das Tor zum Palast zu unserer linken und wir fühlen uns sicher! 
Unsere Zeit ist gekommen.
Überquere die Straße und gehe nach links durch das Tor. \todo[noline]{Thinking about which group arrives first}
Lass die staubigen Straßen hinter dir, du bist schon lang genug auf ihnen gelaufen. 
Es wird Zeit!
Gehe durch das Tor und betrete den Palast.
Bist du jetzt froh mir gefolgt zu sein? \\

Durchschreite das Foyer und gehe durch ein weiteres Tor. \todo[noline]{Evtl. empfiehlt es sich für Begin und Ende den gleichen Raum, aber für diese Szene einen anderen zu verwenden.}
Siehst du, wie links und rechts die Menschen stehen dir zuzujubeln?
Sie begrüßen dich. Dich! Sie sind wegen dir hierher gekommen.
Gehe auch durch das zweite Tor und du betrittst den großen Saal. 
Hier stehen zwei mal zwei Reihen von Stühlen. 
Eine dieser Reihen steht auch der Tribüne. 
Dort ist dein Platz. 
Nehme Platz auf einem der Stühle auf der Tribüne.

\subsection{Die Ansprache}
Du bist noch da, du kannst mich noch hören.
Wer bin ich für dich jetzt geworden, nach der Zeit, die wir miteinander verbracht haben?
Nach der Zeit, die ich in deinem Ohr saß.
Doch nun, jetzt endlich, ist \textit{deine} Zeit gekommen. 
Hast du den Zettel noch in der Hand, den unsere Kontaktperson dir gegeben hat?
Siehst du dir Nummer, die auf ihm steht?
Wenn deine Nummer aufgerufen wird, gehört der Saal ganz dir.
Alle werden zuhören und du darfst sprechen. 
Stehe dann von deinem Stuhl auf und spreche zu deinem neuen Volk, zu dem Volk, das nun dein ist,
das zu dir aufschaut den Satz, der auf deinem Zettel steht.
Falls du nicht sprechen magst, kein Sorge, jeder neue Herrscher ist am Beginn noch Nervös.
Es ist eine Vornehmen-Nervosität. 
Es wird dann die Stimme der Gemeinschaft für dich sprechen.


\subsection{Weg zur Eroberung}
\textit{Die Teilnehmer*innen laufen die Strecke zurück zum ersten Gebäude, zu dem Raum mit den Puppen. Der Weg hier ist beschrieben, als einmal um das Gebäude herum.}\\

Wir sind in Stimmung! 
Spürst du es auch, wie dein Blut kocht?
Wir haben es geschafft, die Ansprache war ein Erfolg! 
Wir sind beisammen und jetzt gemeinsam: los! Auf! Die Eroberung beginnt!\\
Verlasse schnell den Saal und begebe dich wieder auf die Straße. 
Achte darauf, dass wir zusammen bleiben!

\begin{itemize}\em
\item[] Aber eines haben sie vergessen:
    dieser Staat, dieser Vornehmen-Organismus, wer trägt ihn? 
    Der zauberhafteste, vornehmste, ideellste Mensch, welches ist seine ökonomische Basis?
    Das ist, was all die Vornehmen-Denker vergaßen. 
\end{itemize}

Und nun nach rechts um das Gebäude herum. 
Unser Ziel ist nahe. 
Nun gilt es zusammenbleiben, nichts übereilen, nicht die Ruhe verlieren, selbst wenn die Gemüter kochend sind.

\begin{itemize}\em
\item[] Es gibt nicht, schicke Kleider, gutes Essen, gute Musik, Bälle und Geigen und Rosen, ohne diejenige, die die Kleider nähen, die Musik spielen.
    Es gibt keine Samtpfoten ohne Schwielen an der Händen.
    Und wer erbt die Samtpfoten, wer die Schwielen?
    Es muss eine arbeitende Klasse geben, die den anderen den Reichtum schenkt.
    Welches ist dann euer Schenker-Prinzip?
\end{itemize}

Hier, jetzt, erneut rechts! 
Gleich sind wir am Ziel. 
Lange muss eure Geduld nicht mehr währen, lange braucht ihr nicht mehr an euch halten!
Eure Triebe dürfen sich jetzt bald schon entladen. 
Lauft rechts und dann weiter die Straße entlang.

\begin{itemize}\em
\item[] Entsinnen wir uns des Privateigentums, als erstes Schenker-Prinzip: 
    es ist widerlich! 
    Nach allen Regeln und Gesetzen ist es falsch! 
    Selbst nach eurem Vornehmen-Denken: das Gerede von Geld, das Geschäfte-Machen, all das passt euch doch auch nicht!
    Was euer Ideal ist, ist der Grundrentner, der von seinem Kapital leben kann, so lange er lebt. 
    Doch das ökonomische Gesetz auf dem es steht, mach auch ihn zum Kapitalist werden.
    Und er, der Kapitalist, der Eigentümer, der Fabrikherr: das sind doch auch keine Gestalten!
    Sie sind euch ebenso widerlich, so Nicht-Vornehm, so gemein, wie uns.
\end{itemize}

Noch ein letztes mal rechts, dann, ganz bald sind wir da. 
Lass dein Blut kochen.
Kochendes Blut ist was es braucht. Für uns, für uns Eroberer, für unsere Eroberung, oder, wie wollen wir sie nennen? 
Für unsere \textit{konservative Revolution}!
Noch einmal rechts, die Straße entlang und dann, gleich, sind wir da!

\begin{itemize}\em
\item[] Entsinnen wir uns der Sklaverei, dem zweiten Schenker-Prinzip: 
    die Sklaverei wäre schön, sie wäre gut! 
    Kein gemeines Gerede, kein Spielen um den Wohlstand, hier tanzt und singt das Leben der Vornehmen in voller Kraft! 
    Doch kann es tanzen und singen, wo es nicht geistig ist? 
    Nein, das Problem der Sklaverei ist nicht die Moral, das Problem der Sklaverei ist die Natur!
    Zu viel Natur bekommt euch nicht, ihr edlen Geister.
    Ihr lebt doch, ihr zerrt doch nur von dem Natur-Gegensatz, von dem Geistig-Sein, von dem Geistreich-Sein.
\end{itemize}

Jetzt durch diese kleine Tür in den Vorraum. 
Wir betreten das Haus des Ressentiments. 
Hier sind die letzten Gestalten die Überzeugt und zurechtgemacht werden müssen. 
Sie sind ahnungslos. 
Gleich werden wir sie stürmen. 
Geht in den Vorraum, aber betretet noch nicht den Saal, wartet auf das Zeichen!

\begin{itemize}\em
\item[] Die behauptete Berechtigung der \so{Sklaverei} [...] sowie die Berechtigung einer \so{Herrschaft} als bloßer Herrschaft überhaupt [...] beruht auf dem Standpunkt, den Menschen als \so{Naturwesen} überhaupt nach \so{einer Existenz} [...] zu nehmen, die seinem Begriffe nicht angemessen ist.\footATCite[][77, §57]{hegel_grundlinien_2017}\\
    \emph{Also Nein!} Denn [d]er freie Geist ist eben dies, nicht als der bloße Begriff, oder \so{an sich} zu sein, sondern [...] die unmittelbare natürliche Existenz aufzugeben und sich die Existenz nur als die seinige, als freie Existenz zu geben\footATCite[][77, §57]{hegel_grundlinien_2017}.
\end{itemize}

Wir sind versammelt, diese letzte Tür und die letzte Tat und wir sind frei! 
Und die Stadt ist frei!
Die letzte Tat und diese Stadt ist frei. 
Macht euch bereit! 
Macht euch bereit den letzten Schritt zu gehen. 
Wartet auf den Countdown, dann gehen wir gemeinsam!\\

\begin{itemize}\em
\item[] Also, fortan, ihr hohen Geister, schafft euch die einzige menschenwürdige Existenz, die geistige Existenz!
    Hinfort, mit dem Sklaven-Denken, hinfort mit dem Privateigentum! 
\end{itemize}

10. 9. 8. ...

\begin{itemize}\em
\item[] Behaltet euer Vornehmentum, wir mögen es auch, aber anerkennt, dass das wirkliche Vornehmentum nur in der menschlichen Existenz ist:
    das heißt: Aufhebung des Privateigentums, gemeinschaftliche Arbeit, gemeinschaftlicher Wohlstand, Reduzierung der Arbeitszeit bei gleichzeitiger Erhöhung der Produktivkräfte!
\end{itemize}

7. 6. 5. 4. ...

\begin{itemize}\em
\item[] Das war schon der Fehler, den alle Vornehmen-Denker, alle Nietzsches und was weiß ich nicht gemacht haben: sie haben den Sozialismus gehasst, und deswegen nicht gesehen, dass er es ist, der die einzige mögliche Grundlage für eure Philosophie bildet.
    Also fortan, ihr Betrogenen, auf in die Schlacht um eure falschen Werte!
    Auf in den Kampf, für eure Naturverkehrtheit!
\end{itemize}

3. 2. ...

\begin{itemize}\em
\item[] Es gibt kein Zurück mehr, ihr habt es geboren, was nie mehr geboren werden sollte: die Wiederholung der Geschichte!
    Genießt es, und denkt nicht mehr zu sehr daran.
    Eure Triebe werden nur noch ein Mal so entfesselt sein, dann wird die Geschichte mit euch Abrechnen. 
    Das wird dann \emph{mein} \qq{Tabula Rasa} sein!
    Fortan ihr Betrogenen, voraus und immer weiter ins Getümmel.
\end{itemize}

1! 
Öffnet das Tor! 
Eure Zeit ist gekommen! 
Nehmt euch nun, was euch gehört!


\subsection{Die Eroberung}
\textit{(Die Teilnehmer*innen betreten wieder den ersten Raum. 
Die früheren blutbeschmierten Puppen sind \hyperref[zahl_der_puppen]{teilweise} weggeräumt.
Statt ihrer stehen, liegen, hängen im Raum verteilt unbemalte, nicht blutverschmierte Puppen.
Außerdem stehen Eimer mit Farbe und liegen Pinsel hier und Dort auf dem Boden, zwischen den Puppen umher.)}\\

Wir beginnen nun damit die Stadt auszumalen, wie es uns vorschwebte.
Nimm dir einen Pinsel, trau dich, wenn du magst, nimm deine Hände. 
Bearbeiten die Menschen, die dir nun Untertan geworden sind!\\

\begin{itemize}
    \item[] \textit{Das merkwürdigste und zugleich wichtigste Phänomen der Massenbildung ist nun die bei jedem Einzelnen hervorgerufene Steigerung der Affektivität [...].
    Man kann sagen, meint McDougall, dass die Affekte der Menschen kaum unter anderen Bedingungen zu solcher Höhe anwachsen, wie es in einer Masse geschehen kann, und zwar ist es eine genussreiche Empfindung für die Beteiligten, sich so schrankenlos ihren Leidenschaften hinzugeben und dabei in der Masse aufzugehen, das Gefühl ihrer individuellen Abgrenzung zu verlieren.
    Dies Mitfortgerissenwerden der Individuen erklärt McDougall [...] durch die uns bereits bekannte Gefühlsansteckung. [...]
    Dieser automatische Zwang wird um so stärker, an je mehr Personen er gleichzeitig derselbe Affekt bemerkbar ist.
    Dann schweigt die Kritik des Einzelnen, und er lässt sich in denselben Affekt gleiten. [...]
    die gröberen und einfacheren Gefühlsregungen haben die größere Aussicht, sich auf solche Weise in einer Masse zu verbreiten.\footfullcite[][79]{freud_gesellschaft_2013}}

    \textit{[...] Die Masse macht dem Einzelnen den Eindruck einer unbeschränkten Macht und einer unbesiegbaren Gefahr.
    Sie hat sich für den Augenblick an die Stelle der gesamten menschlichen Gesellschaft gesetz, welche die Trägerin der Autorität ist, deren Strafen man gefürchtet, der zuliebe man sich so viele Hemmungen auferlegt hat.
    [...] Im Gehorsam gegen die neue Autorität darf man sein früheres \glqq Gewissen\grqq{} außer Tätigkeit setzen und dabei der Lockung des Lustgewinns nachgeben. [...]\footATCite[][79 ff.]{freud_gesellschaft_2013}}
\end{itemize}
    

\section{B - Die Bürger}

\subsection{Die erste Entscheidung}

\subsection{Weg zur Tribüne I}

\subsection{Action X}

\subsection{Die Überzeugung}

\subsection{Die Ansprache}
Hört, jetzt seht ihr sie endlich, da sind sie, die Herrscher, auf die du gewartet hast.
Hast du auch keine Zweifel? 

\subsection{Weg zur Eroberung}

\subsection{Die Eroberung}

\section{C - Die Revoluzzer}

\subsection{Das Tribunat}
\textit{Während die anderen den Raum verlassen haben geht langsam das Licht wieder an und beleuchtet die blutbeschmierten Puppen}

\subsection{Der Konflikt}

\subsection{die Überzeugung}

\subsection{Störung der Ansprache (optional)}

\subsection{Die Gleichmachung oder Das Schafott}


\backmatter

\chapter* {Ideen und Konzepte}\label{ideen_konzepte}
\section*{Gruppen}
\begin{itemize}
    \item[A] \textbf{Eroberer} Wollen die Stadt Chili einnehmen.
    \item[B] \textbf{Bürger} Haben von der Ankunft der Eroberer gehört und wollen sie in Empfang nehmen.
    \item[C] \textbf{Revoluzzer} Haben von den Eroberern gehört und wollen diese überführen.
\end{itemize}

\section*{Stationen}
\textbf{A Eroberer}
\begin{enumerate}
    \item Weg zur Tribüne (Location: Weg entlang der Gasse, dann zu Kreuzung und wieder zurück zu Raum 1)
    \item Begegnung mit Gruppe C. (Location: Gasse)
    \item Der Händedruck (Location: Kreuzung \textit{Kantstraße, Sandweg})
    \item Ansprache (Location: 1. Raum auf der Tribüne)
    \item Weg zur Eroberung (Location: Weg einmal um das Gebäude)
    \item Die Eroberung (Location: 1. Raum)
\end{enumerate}
\textbf{B Bürger}
\begin{enumerate}
    \item Weg zur Tribüne (Location: Weg zu Location X und dann entlang der Gasse zurück zu Raum 1)
    \item Action X (Location X)
    \item Überzeugung (Location: Gasse)
    \item Ansprache (Location: 1. Raum vor der Tribüne)
    \item Weg zur Eroberung (Location: Weg einmal um das Gebäude)
    \item Die Eroberung (Location: 1. Raum)
\end{enumerate}
\textbf{C Revoluzzer}
\begin{enumerate}
    \item Das Tribunate - Einzelne Vorträge (Location: 1. Raum)
    \item Der Konflikt (Location: Gasse)
    \item Überzeugung (Location: Gasse)
    \item Evtl. Störung der Ansprache (Location: 1. Raum)
    \item Die Gleichmachung oder Das Schafott  (Location: 1. Raum)
\end{enumerate}

\section*{Brainstorm}

\subsection*{Der Zettel}
In \glqq Die Gasse\grqq{} erhalten die Spieler*innen (Gruppe A + C) kurz Zeit sich zu unterhalten.
Hier können die Spieler*innen aus C denen aus A das Codewort mitteilen:
\glqq Sagt zu dem Mann, dem ihr die Hände geben sollt: \glq \textit{Dir} gebe ich nicht die Hand!\grq\grqq.
Sagen die Spieler*innen tatsächlich \glqq Dir geben ich nicht die Hand\grqq{} erhalten die Spieler*innen statt dem Zettel mit dem Text für die Ansprache einen Zettel mit dem QR-Code zu einer neuen Gruppe.
Falls Spieler*innen Teil dieser Gruppe werden, weiß C von der Location der Ansprache.
Dann findet \qq{Die Störung der Ansprache} statt.

\subsection*{Reden}
Es ist zweifelhaft, ob man den Teilnehmer*innen zutrauen sollte zu reden, oder ob dies komische Momente, wo sich die Teilnehmer*innen evtl. auf ungute Weise unwohl fühlen, erzeugt.
Alternativ müsste auch dies über die Kopfhörer laufen. 
Ggf. Könnte alle aber zunächst als von den Teilnehmer*innen gesprochen angesehen werden, Alternativen müssten diskutiert werden, bzw. evtl. auf das Publikum angepasst werden. 
Generell ist hier das Problem, dass die Inszinierung (auch) auf dem Prinzip der Überraschung funktioniert: 
die Teilnehmer*innen wissen noch nicht, was sie erwartet.

\subsection*{Zahl der Puppen}\label{zahl_der_puppen}
Evtl. könnte die Anzahl der weggeräumten, blutbeschmierten Puppen abhängen von der Zahl der Menschen, die sich letztendlich den Revoluzzern angehängt hat. 
Oder generell der zahl an Menschen, die ihre Spur gewechselt haben.
Es wäre dann noch die Frage welcher Fall für mehr oder weniger Puppen sorgt...

\subsection*{Die Reden der Gruppe A}
Eine Person könnte Zizek-like so ganz offensichtlich das Motiv der Rechten präsentieren, indem er klar sagt, unsere Werte: Christentum, Gott, etc., etc., aber dahinter steht, dass du deine Trieben ausleben darfst.

\subsection*{Aufteilung Gruppe A und B}
Alle Menschen die den ersten link wählen bekommen abwechselnd die Spur von Gruppe A und B zugeteilt.

\subsection*{Stimmverteilung}
Für Gruppe A sollte die Stimme mit den Einwürfen (Eingerückte Stellen) nach und nach zur Sympathie-Trägerin werden.
Sie enthält von Beginn an die kritischen Einwürfe. 
Während im Gegensatz dazu in Gruppe C sich vor allem mit ihrer Hauptstimme identifizieren soll, um den Bruch der scheinbar positiven Ideologie zu verhärten.

\chapter*{Technische Umsetzung}\label{technische_umsetzung}

   
\printbibliography

\listoftodos
 
\end{document}
